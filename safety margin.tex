\documentclass[review]{elsarticle}
%\documentclass{elsarticle}

\usepackage{lineno,hyperref}
\modulolinenumbers[5]

\usepackage[printonlyused]{acronym}
\usepackage{comment}
\usepackage{amssymb}
\usepackage[fleqn]{amsmath}
\usepackage{mathtools}
\usepackage[version=3]{mhchem}
\usepackage[labelfont=bf,font={footnotesize}]{caption}
\usepackage[labelfont=bf,font={footnotesize}]{subcaption}
\usepackage{color}

\usepackage{amsthm,gensymb}
\usepackage{longtable}

\usepackage{booktabs,listings,soul,multirow,comment}

\journal{TBD}

%%%%%%%%%%%%%%%%%%%%%%%
%% Elsevier bibliography styles
%%%%%%%%%%%%%%%%%%%%%%%
%% To change the style, put a % in front of the second line of the current style and
%% remove the % from the second line of the style you would like to use.
%%%%%%%%%%%%%%%%%%%%%%%

%% Numbered
%\bibliographystyle{model1-num-names}

%% Numbered without titles
%\bibliographystyle{model1a-num-names}

%% Harvard
\bibliographystyle{kee-nelson-wortman/bibliography/model2-names.bst}\biboptions{authoryear}

%% Vancouver numbered
%\usepackage{numcompress}\bibliographystyle{model3-num-names}

%% Vancouver name/year
%\usepackage{numcompress}\bibliographystyle{model4-names}\biboptions{authoryear}

%% APA style
%\bibliographystyle{model5-names}\biboptions{authoryear}

%% AMA style
%\usepackage{numcompress}\bibliographystyle{model6-num-names}

%% `Elsevier LaTeX' style
%\bibliographystyle{elsarticle-num}
%%%%%%%%%%%%%%%%%%%%%%%

%% The ridiculous Nuclear Technology Citation
%\bibliographystyle{ieeetr}

%\renewcommand\thetable{\Roman{table}} %Roman numeral table numbers!!!

\usepackage{cleveref}

\input{acronym}
%

\begin{document}
\begin{frontmatter}

\title{TBD}
%\tnotetext[mytitlenote]{Fully documented templates are available in the elsarticle package on \href{http://www.ctan.org/tex-archive/macros/latex/contrib/elsarticle}{CTAN}.}

%% Group authors per affiliation:
%\author{Elsevier\fnref{myfootnote}}
%\address{Radarweg 29, Amsterdam}
%\fntext[myfootnote]{Since 1880.}

%% or include affiliations in footnotes:
\author[a]{\corref{mycorrespondingauthor}}
%\ead[url]{www.tamu.edu}
\cortext[mycorrespondingauthor]{Corresponding author}
\ead{}

\author[a]{}


%\address[mymainaddress]{3100 TAMU, College Station, TX 77483}
%\address[mysecondaryaddress]{Austin, TX}
%\address[c]{Birmingham, AL}
%\address[d]{Champaign, IL}


\address[a]{TBD%, erniekee@tamu.edu
}
\begin{comment}
\begin{abstract}
A quantifiable method to obtain an ordered ranking of alternative maintenance policies in a simple system is summarized.
Application of the ranking method to economic analysis of a regulated process is described.
A new model for the impact of regulation and level of care on maintenance policy decision-making is
presented.
\end{abstract}

\begin{keyword}
RISMC \sep Maintenance policy \sep Decision analysis
\end{keyword}
\end{comment}

\end{frontmatter}

\section{Example}\label{sec:introduction}
	We evaluate alternative designs for a simple, unreliable protection system shown in 
	\Cref{fig:tuncsystem_reduced} designed to pump water in order to prevent catastrophic failure. 
	Such systems are commonly used in many industrial processes and are frequently under
	regulation; Standard For The Inspection, Testing, And Maintenance of Water-Based Fire Protection 
	Systems, NFPA 25 and Domestic Licensing of Production and Utilization Facilities, 10 CFR Part 50 are
	examples of a standard typically adopted in regulation and US federal regulation.
	We assume a profit-making enterprise that operates for public good under regulation operates the 
	simple system of \Cref{fig:tuncsystem_reduced} for regulatory compliance (for example, fire suppression).
	
	\begin{figure}[ht]
	\centering
	\includegraphics[width=0.7\textwidth]{kee-nelson-wortman/graphics/tuncsystem_reduced.png}
	\caption{Illustration of a system simplified from \citet{t2013}The system normally operates to maintain 
	the level in the tank with the fish at `x' based on output from L1 by intermittently  operating the pump 
	P2 between the levels $\alpha_1$ and $\alpha_2$. L1 causes P2 to  start when level reduces to 
	$\alpha_1$ and stop when level reaches $\alpha_2$. At level  $\tilde{a}$, P1 stops, and at level 
	$\tilde{b}$, P2 stops.}. 
	\label{fig:tuncsystem_reduced}
	\end{figure}
	
	We assume different maintenance policies can be adopted in order to avoid different catastrophes; 
	property loss, pollution, wildlife injury or death, injuries to human life from illness to death.
	A fundamental premise is that such alternative maintenance policies are available and
	can be effected at different cost with different levels of system reliability.

\begin{comment}
	We assume a time-dependent market price for the product the system supports and a
	time-independent level of production.
	That is, there exists an ongoing competitively priced market for the product but any 
	maintenance policy chosen will not change the level of production (activity); however 
	others in the market similarly unable to change production levels can affect the 
	market price over time.
		
	In general, the cost for nuclear power equipment maintenance, in the absence of
	subsidies (such as regulated electricity markets, or subsidized `green energy') has higher cost 
	(due to greater level of care) than for other electricity production models.
	We assume a desire to investigate alternative maintenance polices against high cost of
	maintenance motivated by improved profit.
	Finally, because we will later describe loss models, we assume the expected loss for any
	given maintenance policy is fixed.

	An example of the activity we describe under the assumptions above are closely related
	to a merchant nuclear power plant operated in the United States national electrical grid.
	These plants have a nearly constant cost of production and to cover cost of production, 
	operate continuously at maximum power output.
	However, the nuclear production level represents only about 20\% of total electrical production
	on the national grid.
	Almost all of the remaining the electrical grid demand is satisfied by coal, gas, and oil plants.
	Because fossil fuel prices swing a great deal over time (primarily natural gas), about 80\% of the electricity market 
	reflects those prices; market price of nuclear power production must follow (at least on downswings) 
	to some extent in order to stay in the market at the 20\% level.
			
	We generally follow an approach \citet{8993548820130601} recommends, but extend 
	his method %to the \ac{rismc} domain and assert that 
	to include formal risk analysis, as is done in \ac{pra}, 
	for cost of due care desired in the negligence rule as described by \citeauthor{8993548820130601}.
	For now, we ignore contributory negligence such as public disregard of warning sirens or
	notification of impending danger due to system failures provided by the system operator.
	It is clear shared liability is indicated; the public should maintain a reasonable evacuation 
	capability and response when clear warnings are given \citep[for example][in nulcear accidents]{edselc.2-52.0-002243974919860101} and \citep[for example][in a consequential response to volcano warnings]{469228419860302,2W6257764379020150518}.
	We believe, under the assumptions taken in here, the level of care asked for in liability 
	evaluation can be related (through \ac{pra}) to successful accident avoidance associated 
	with the maintenance policy chosen by the enterprise.

\end{comment}
%	In particular, we follow Shavell's simplification that the cost of verifying compliance with 
%	regulation is incurred for sure but the cost of evaluating compliance is experienced 
%	only with the probability that harm occurs. 
%	As he asserts, the negligence rule consequentially involves lower expected enforcement 
%	costs than regulation.
	
%\input{paul}

\subsection{Analysis of system design}\label{sec:analysis}
	We analyze the system operation as follows.
	When a perturbation (initiating event such as fire or a need for supplemental cooling water) occurs, 
	P1 must start up and deliver water at design pressure.
	As P1 delivers water, the tank level, $x$, decreases until it reaches $\alpha(1)$.
	At $\alpha(1)$, L1 sends a signal to actuate P2 to refill the tank until the level reaches
	$\alpha(2)$.
	In our analysis, if any equipment fails, P1, P2 or L1, then the system is assumed to fail 
	catastrophically.
	
	To maintain clarity and reduce complexity, we ignore many aspects that must 
	be considered in an actual system design evaluation where catastrophic failure would cause harm 
	to the public.
	Coincident failures may need to be taken into account; the triggering event may cause multiple 
	protection system failures or system failures may occur independently from different causes.
	Consideration of public disregard or failure of warning sirens or notifications of impending danger 
	should be included.
	Of course, some public burden must be assumed; the public should maintain a reasonable evacuation 
	capability and response when clear warnings are given \citep[for example][in nuclear accidents]{edselc.2-52.0-002243974919860101} and \citep[for example][in a consequential response to volcano warnings]{469228419860302,2W6257764379020150518}.
	
%	We are assuming the negligence rule applies and therefore at a certain level of care, the
%	enterprise (injurer) is (theoretically) relieved of liability.
%	Victims of certain nuclear accidents would be compensated through private insurance
%	up to a required amount \citep{pa1957,9939787020141101}.
%	In here we take the insurance cost to be a sunk cost and exclude it from further consideration.
	
\subsection{Alternatives}\label{sec:alternatives}
	Now we assume an engineer is responsible for the performance of P1, P2 and L1.
	She is tasked with defining maintenance policy that includes safety margin against failure.
	%We recognize the system is designed unreliably given the possible consequences for failure however
	%we assume the design for the purposes of illustration.
	That is, the only maintenance policy available to the engineer is modifications with associated cost 
	implications that result in different failure probabilities that change with safety margin of the three 
	devices included in the system.
	
	We assume the alternatives, (\Cref{tab:alternatives} column Alt.), enumerated in \Cref{tab:alternatives} with 
	maintenance policy costs and failure probabilities for each Alternative are; maintenance cost $A(\cdot)$,  
	failure probability $a(\cdot)$; maintenance cost $B(\cdot)$, failure probability $b(\cdot)$; and maintenance cost $C(\cdot)$
	has failure probability $c(\cdot)$.
	For now, we assume the costs are known exactly, leaving only the reliability of the maintenance policy uncertain.
				
	\begin{table}[ht]
	\caption{Alternative (Alt.) maintenance policies, \{$a(\cdot)$,$b(\cdot)$,$c(\cdot)$\}, available to the engineer. 
	Each alternative policy has a different associated cost.}
	\centering
	\begin{tabular}{l lll | l lll | l lll} \\
	\toprule
	Alt. & P1		&	L1		&	P2		& Alt. & P1		&	L1		&	P2		
	& Alt. &  P1		&	L1		&	P2		\\ \midrule
		
	0 &		$a(0)$		&	$b(0)$		&	$c(0)$			&	12 &		$a(1)$	&	$b(1)$	&	$c(0)$		
	&	24 &	$a(2)$	&	$b(2)$	&	$c(0)$	 \\

	1 &		$a(0)$		&	$b(0)$		&	$c(1)$		&	13 &		$a(1)$	&	$b(1)$	&	$c(1)$	
	&	25 &		$a(2)$	&	$b(2)$	&	$c(1)$ \\

	2 &		$a(0)$		&	$b(0)$		&	$c(2)$		&	14 &		$a(1)$	&	$b(1)$	&	$c(2)$	
	&	26 &		$a(2)$	&	$b(2)$	&	$c(2)$ \\
		
	3 &		$a(0)$		&	$b(1)$	&	$c(0)$			&	15 &		$a(1)$	&	$b(2)$	&	$c(0)$		& & & \\
	4 &		$a(0)$		&	$b(1)$	&	$c(1)$		&	16 &		$a(1)$	&	$b(2)$	&	$c(1)$	& & & \\

	5 &		$a(0)$		&	$b(1)$	&	$c(2)$		&	17 &		$a(1)$	&	$b(2)$	&	$c(2)$	& & & \\

	6 &		$a(0)$		&	$b(2)$	&	$c(0)$			&	18 &		$a(2)$	&	$b(0)$		&	$c(0)$		& & & \\

	7 &		$a(0)$		&	$b(2)$	&	$c(1)$		&	19 &		$a(2)$	&	$b(0)$		&	$c(1)$	& & & \\

	8 &		$a(0)$		&	$b(2)$	&	$c(2)$		&	20 &		$a(2)$	&	$b(0)$		&	$c(2)$	& & & \\

	9 &		$a(1)$	&	$b(0)$		&	$c(0)$		&	21 &		$a(2)$	&	$b(1)$	&	$c(0)$		& & & \\
		
	10 &		$a(1)$	&	$b(0)$		&	$c(1)$	&	22 &		$a(2)$	&	$b(1)$	&	$c(1)$	& & & \\

	11 &		$a(1)$	&	$b(0)$		&	$c(2)$	&	23 &		$a(2)$	&	$b(1)$	&	$c(2)$	& & & \\ \bottomrule
	\end{tabular}
	\label{tab:alternatives}
	\end{table}%

	\subsection{Distribution on alternatives}\label{sec:distribution_on_alternatives}
		A naive failure logic in event tree format is shown in \Cref{fig:tuncsystem_reduced_event_tree}.
		For purposes of illustration, we accept for now the simplicity implied by this logic and leave more
		complete analysis to future work \citep[see][for example]{zkrkm2013,t2013}.
	
		\begin{figure}[ht!]
		\centering
		\includegraphics[width=0.9\textwidth]{kee-nelson-wortman/graphics/tuncsystem_reduced_event_tree.png}
		\caption{Illustration of the event tree logic for the system, assuming P1 has failure probability a,
		L1 has failure probability b, and P2 has failure probability c and showing all possible outcomes.}
		\label{fig:tuncsystem_reduced_event_tree}
		\end{figure}

		Referring to \Cref{fig:tuncsystem_reduced_event_tree}, each alternative's success probability 
		could be worked out from the event tree solution. 
		Further, we assume the failure probabilities are uncertain and the engineer knows
		the distribution for a basic design \citep[][for example]{ewgra2007}.
		For example, Alternative~3 would work out to be:
		\begin{equation}\label{eq:event_tree_solution}
		p[SUCCESS|I,Alt(3)] = [1-a(0)][1-b(1)][1-c(0)] \nonumber
		\end{equation}		
		\noindent and the cost, $\mathbb{C}(3)$, would work out to be	
		\begin{equation}
		\mathbb{C}(3) = A(0) + B(1) + C(0). \nonumber
		\end{equation}
		\noindent By assumption, the distribution on $p[SUCCESS|I,Alt(3)]$ is only
		dependent on the maintenance policy assigned (known by the engineer).		

		\begin{table}[ht]
		\caption{Level transmitter (L1), turbine-driven pump (P1), and motor driven pump (P2) probability distributions
		for 24 hours of operation and assuming rare event.}
		\centering
		\begin{tabular}{l c c c l}
			& \multicolumn{3}{c}{NUREG 6928 Failure to run parameters} &				\\
			& Mean		& $\alpha$ 	& $\beta$ 		& Failure probability, 24 hr		\\
		L1	& 1.02E-07	& 0.5			& 4.92E+06	&  5.19E-05 				\\
		P1	& 5.77E-06	& 3.422		& 5.93E+05	&  4.39E-03 				\\
		P2	& 4.54E-06	& 1.655		& 3.65E+05	&  8.65E-05				\\
		\end{tabular}
		\label{tab:failure_parameters}
		\end{table}%


\begin{comment}
		
	\subsection{The level of care}\label{sec:the_level_of_care}
		In our analysis, we assume the level of care taken by the engineer in order to avoid consequential
		failures is the same as the success probability (SUCCESS in \Cref{fig:tuncsystem_reduced_event_tree}.
		That is, by selecting maintenance policies that maximize $p[SUCCESS|I,Alt(\cdot)]$, she exercises
		the maximum level of care.
		
		For now, it is useful to assume that the consequences of failure don't include internal costs of repair or losses
		associated with destruction of the system such as burned equipment due to inadequate cooling
		flow or overfilling of the tank causing overload breakage.
		Instead we assume liabilities external to the system such as property damage, injury, and so forth as 
		mentioned previously.
		
	\subsection{The utility}\label{sec:the_utility}			 
		The utility, $u$, is assumed to be maximized when total cost is minimized; and the costs are constrained
		to include only liability cost, $\ell(\cdot)$, and the cost of maintenance (\Cref{sec:the_level_of_care}).
		Under the negligence rule beyond a certain level of care exercised, say $X^*$, the enterprise (in this case) would not
		have liability for consequences.
		
		\subsubsection{Level of care}\label{sec:the_level_of_care}
			We propose the level of care is entirely defined by $p[SUCCESS|I,Alt(\cdot)]$ and therefore the cost
			at any level of care is $X[Alt(\cdot)]$.
			Given 
%			The cost function for the level of care is therefore
%			\begin{equation}
%			\begin{split}
%			X(j) & = \mathbb{C}(j);\, \text{for} \, j \le j^*, \\
%			X(j) & = 0.0;\, \text{for} \, j > j^*,
%			\end{split}
%			\end{equation}			
%			\noindent where $j$ is the index on $p[SUCCESS|I,Alt(j)]$ ordered such that
%			\begin{equation}
%			\begin{split}
%			 & p[SUCCESS|I,Alt(0)] \ge p[SUCCESS|I,Alt(1)] \ge   \nonumber \\
%			& p[SUCCESS|I,Alt(2)] \ge \,  \ldots \, \ge p[SUCCESS|I,Alt(26)].  \nonumber
%			\end{split}
%			\end{equation}
			 
%		\subsubsection{The loss}\label{sec:the_loss}
			\noindent We assume the loss, $\ell(\cdot)$, associated with a (consequential) system failure 
			decreases monotonically with level of care as illustrated in \Cref{fig:utility} \citep[][]{8993548820130601}.					
			\begin{figure}[ht]
			\centering
			\includegraphics[width=0.9\textwidth]{kee-nelson-wortman/graphics/utility.png}
			\caption{Conceptual illustration of the total accident cost assuming the level of care
			is defined by the failure probability.  Shown in the figure is the point in level of
			care at the optimal level, $X^*$,beyond which no further liability is realized.}
			\label{fig:utility}
			\end{figure}	
			\noindent Although there exists a unique value for $X^*$ for the case shown in 
			\Cref{fig:utility}, it is possible that $X^*$ may not be unique and may not exist.
			That is, there are no restrictions on $C(\cdot)$ or $\ell(\cdot)$ that would force
			existence and uniqueness.
			We can say that, given $X^*$ is not unique, a profit-making enterprise would 
			choose the first $X^*$ (lowest cost) since the accident cost may never be realized.
			\textcolor{blue}{I wonder if this is true? It could be a choice for least damage as well ...}
			
			The total accident cost, $\mathbb{T}(j)$, under maintenance policy $Alt(j)$ will be
			limited by the policy $Alt(j^*)$ where the appropriate level of care is met,
			\begin{equation}
			\begin{split}
			\mathbb{T}(j) = X(j) + \ell \, \overline{\{p[SUCCESS|I,Alt(26)]\}},
			\end{split}
			\end{equation}
			\noindent and the costs, $X(\cdot)$ and $\ell(\cdot)$ are taken to be positive so
			where they appear with positive values (revenue) they will be subtracted.

	\subsection{Profit}
		Under the assumptions and definitions we have, the utility function for the level
		of care considering maintenance policy would indicate profit $\mathbb{P}(\cdot)$ assuming
		revenue $\mathbb{R}(T)$ over some time period, $T$,
		\begin{equation}\label{eq:profit_equation_certain}		
		\mathbb{P}(T,j) = \begin{cases} \mathbb{R}(T) - X(j^*)-\ell \,\overline{\{p[SUCCESS|I,Alt(j^*)]\}} & \mbox{if} \; j \ge j^*    \\
					\mathbb{R}(T) - X(j)-\ell \,\overline{\{p[SUCCESS|I,Alt(j)]\}} & \mbox{if} \;j < j^* \end{cases}
		\end{equation}
		\noindent where losses (as shown in the figure) are taken to be positive.
		
	\subsection{On the behavior}
		It isn't necessary that the level of care cost function, as we define it, to behave such that cost 
		strictly increases with increasing level of care.
%		In fact, enterprises engaged in long term process improvement may prioritize projects in such a way
%		to maximize spending under uncertainty \citep[][for example]{4908616120091001}.
		That is, one could imagine a new design for a replacement part that has greater reliability
		at much lower cost.
		When the engineer selects that product in her maintenance policy, the level of care (as we
		define it) is higher but at lower cost than for a policy that doesn't include the new design.
		We assume \citep{pa1957}, $\mathbb{P}(T,j)$ is maximized when the level of care is at $j^*$
		so the enterprise would normally choose to operate on the maintenance policy corresponding 
		to $j^*$.
		If cases are observed where $j^*$ is not uniquely determined, one might expect the enterprise to adopt
		the least cost policy.
	
\section{Preference ordering}\label{sec:preference_ordering}
	We previously assumed the engineer can know the reliability and cost for the maintenance policy
	she adopts.
	However, it is unlikely she knows either of these parameters exactly; she may only know a range of
	likely performance and cost.
	Also, knowing $\ell(\cdot)$ and $\mathbb{R}(\cdot)$ is likely to be even more challenging than 
	the equipment-related parameters.
	Because of this, finding $j^*$ is challenging and fraught with uncertainty.
	
	However, it might be reasonable to ask the engineer (in the absence of more detailed information)
	if she can at least give a ranges of values for $E\{\mathbb{R}(T)\}$; $E\{A(j)\}$, $E\{B(j)\}$, $E\{C(j)\}$;
	$E\{\ell\{p(j)\}$; $E\{a(j)\}$, $b\{A(j)\}$, $E\{e(j)\}$.
	With these expected value ranges, we would be able to develop an uncertainty distribution, $\mathcal{F}(\cdot)$,
	on $\mathbb{P}(\cdot)$.	
	With $\mathcal{F}(\cdot)$ in hand, a normative ordering for the utility, $\mathbb{P}(T,i)$ can be effected as:
	\begin{equation}\label{eq:utility_ordering}
	\int_{l_{max}(i)}^{l_{min}(i)}{\mathbb{P}(T,i) d \mathcal{F}(\mathbb{P}(T,i)) } \ge \int_{l_{max}(i-1)}^{l_{min}(i-1)}
	{\mathbb{P}(T,i-1) d \mathcal{F}(\mathbb{P}(T,i-1)) }
	\end{equation}	
	\noindent where, depending on $\mathcal{F}(\mathbb{P}(T,i-1))$, the uncertainty could shift the ordering from 
	\eqref{eq:profit_equation_certain}.
%	\begin{figure}[ht]
%	\centering
%	\includegraphics[width=0.9\textwidth]{graphics/utility_function.png}
%	\caption{Illustration of a possible derivation of the utility function $u(\cdot)$ from the event tree 
%	solution for on the distribution of $p()$ (having quantiles $q$ and $\tilde{q}$).}
%	\label{fig:utility_function}
%	\end{figure}
	
%	\subsection{Cost}\label{sec:cost}
%		Our economic investigation considers four entities enumerated below 
%		having interest in the outcomes of consequential system failures.
		
%		\begin{enumerate}
%		\item The owner-operator
%		\item The engineer
%		\item The regulator
%		\item The courts
%		\end{enumerate}

\section{Conclusions}\label{sec:conclusions}
	An event tree structure can be used to find alternative costs as distributions which, in turn
	are useful as definitive cost models in Shavell's stylized model.
	
	
	
\end{comment}
	
%	\bibliographystyle{chicago}
	\bibliography{kee-nelson-wortman/bibliography/safety_margin.bib}

\end{document}  